\documentclass[11pt, oneside]{memoir}

\usepackage{fullpage}
\usepackage{xspace}
\usepackage{makeidx}
\usepackage{listings}
\usepackage[colorlinks=true, linkcolor=blue]{hyperref}

\setlength{\parindent}{0pt}
\nonzeroparskip

% add padding to ctabular tables
\renewcommand{\arraystretch}{1.2}

\makeindex

%
% COMMANDS
%
\newcommand*{\cqueues}[0]{\texttt{cqueues}\xspace}
\newcommand*{\key}[1]{#1\index{#1}\xspace}
\newcommand*{\syscall}[1]{\texttt{#1}\xspace}
\newcommand*{\routine}[1]{\texttt{#1}\xspace}
\newcommand*{\fn}[1]{\texttt{#1}\xspace}
\newcommand*{\method}[1]{\texttt{#1}\xspace}
\newcommand*{\module}[1]{\texttt{#1}\xspace}
\newcommand*{\errno}[1]{\texttt{#1}\xspace}

%
% ENVIRONMENTS
%
\lstnewenvironment{code}[1]{
	\lstset{language=#1}
}{
}

\begin{document}

\title{\cqueues userguide}
\date{\today}
\author{William Ahern}
\maketitle

\maxtocdepth{subsubsection}
\setsecnumdepth{subsection}
\pagenumbering{roman}
\tableofcontents
\clearpage
\setcounter{page}{1}
\pagenumbering{arabic}

\chapter{Dependencies}

\section{Operating Systems}

\cqueues heavily relies on a modern POSIX environment. But the fundamental premise is to build on the new but non-standard polling facilities provided by contemporary Unix environments. Specifically, BSD \syscall{kqueue}, Linux \syscall{epoll}, and Solaris Event Ports.

\cqueues should work on recent versions of Linux, OS X, Solaris, NetBSD, FreeBSD, OpenBSD, and derivatives. The only other possible candidate is AIX, if and when support for AIX's \syscall{pollset} interface is added to the embedded ``kpoll'' library.

\subsection{$\lnot$ Microsoft Windows}

Microsoft Windows support is basically out of the question, for far too many reasons to put here. Aside from the more technical reasons, Windows I/O and networking programming interfaces have a fundamentally different character than on Unix. Unix historically relies on readiness polling, while Windows uses event completion callbacks. There are strengths and weaknesses to each approach. Trying to paper over the chasm between the two approaches invariably results in a framework with the strengths of neither and the weaknesses of both. The purpose of \cqueues is to leverage the strengths of polling as well as address the weaknesses.

\section{Libraries}

\subsection{Lua 5.2}
\cqueues requires Lua 5.2. It's not fully portable to Lua 5.1 because it relies on ephemeron tables to prevent coroutine/controller reference cycles. Nonetheless, Lua 5.1 and LuaJIT patches are welcome if not too intrusive, or if they provide a clean refactoring of code.

\subsection{OpenSSL}
The \cqueues sockets module provides seamless SSL/TLS support using OpenSSL.
In the future a \module{cqueues.ssl} module will be provided with OpenSSL bindings for certificate and key management.

\subsection{pthreads}

\cqueues provides an optional threading module, using POSIX threads.\footnote{Building without threading enabled is not well tested.} Internally, it also strives to use thread-safe routines when built with either the \_REENTRANT or \_THREAD\_SAFE macros defined. Thread support is enabled by default 

\paragraph{Linking}
Note that on some systems, such as NetBSD and FreeBSD, the loading application must be linked against pthreads (using -lpthread or -pthread). It is not enough for the cqueues module to pull in the dependency at load time. In particular, if using the stock Lua interpreter, it must have been linked against pthreads at build time. Add the appropriate linker flag to MYLIBS in lua-5.2.x/src/Makefile.

\paragraph{OpenBSD}
OpenBSD 5.1 threading is completely fubar, especially with regard to signals. If using OpenBSD, be sure to compile \emph{without} the thread-safe macros predefined.

\section{Compilers}

The source code is mostly ISO C99 compliant, and even more so with regards to ISO C11. Regardless, it aims to build cleanly with the native compiler for each targeted platform. It currently builds with recent versions of GCC, clang, and SunPro.

GCC and especially clang may emit copious warnings about initializers and unused parameters. These warnings are stupid, particularly the former. The Makefile uses -Wno-override-init (GCC), -Wno-initializer-overrides (clang), and -Wno-unused to quiet these.

SunPro has a broken diagnostic pass confused by macro-generated function calls.\footnote{\texttt{e.g.\ "lib/fifo.h", line 348:\ warning:\ argument mismatch}.} Likewise, the code flow analysis in older versions of GCC---especially on the BSDs---emit erroneous uninitialized use warnings. Newer versions also get this wrong, but usually the code is refactored to reduce noise so legitimate warnings are not suppressed.

For other warnings or errors, patches are welcome.

\section{Makefile}

The Makefile requires GNU Make, usually installed as gmake on platforms other than Linux or OS X.

The Makefile expects Lua 5.2 headers to exist in \$(includedir)/lua/5.2. This is unfortunately not the default install location for Lua headers, although it quite obviously ought to be. To source the usual location, try something like

\begin{code}{sh}
        make prefix=/opt/lua52 lua52include=/opt/lua52/include
\end{code}

\key{M4} and \key{awk} are required to generate errno.c, which also relies on mk/errno.ls and mk/macros.ls to enumerate the system error macro names.

\chapter{Usage}

\section{Conventions}


\subsection{Polling}

\cqueues works through a simple protocol. When a coroutine yields to its parent \cqueues controller, it can pass one or more objects. These objects are introspected for three methods: \method{:pollfd}, \method{:events}, and \method{:timeout}. These methods generate the parameters for installing descriptor and timeout events. When one of these events fires, \cqueues will resume the coroutine, passing the relevant objects which were interested in the triggered event. It's analogous to calling \syscall{poll}, and in fact the routine \routine{cqueues.poll} is provided as a wrapper for \routine{coroutine.yield}.\footnote{In the future this wrapper may be able to detect if the current coroutine was resumed by a controller, and if not chain yield calls until a controller is reached.}

\subsubsection[\method{object:pollfd}]{\method{:pollfd()}} The \method{:pollfd} method should return a descriptor integer or nil. This descriptor must remain in existence until the owner object is garbage collected, \routine{cqueues.cancel} is used, the coroutine executes one additional yield/resume cycle (so the old descriptor is expired from the descriptor queue), or until after the coroutine exits. If the descriptor is closed prematurely, the kernel will remove it from the internal descriptor queue, bringing it out of sync with the controller, and probably causing \method{cqueues:step} to return EBADF or ENOENT errors.

\subsubsection[\method{object:events}]{\method{:events()}} The \method{:events} method should return a string or nil. \cqueues searches the string for the flags `r' and `w', which describe the events to associate with the descriptor---respectively, POLLIN and POLLOUT.

\subsubsection[\method{object:timeout}]{\method{:timeout()}} The \method{:timeout} should return a number or nil. This schedules an independent timeout event. To effect a simple one second timeout, you can do

\begin{code}{perl}
        cqueues.poll({ timeout = function() return 1.0 end })
\end{code}

which is equivalent to

\begin{code}{perl}
	coroutine.yield({ timeout = function() return 1.0 end })
\end{code}

but not quite as simple as the shortcut

\begin{code}{perl}
	cqueues.poll(1.0)
\end{code}

Instantiated \cqueues objects implement all three methods.\footnote{\method{:pollfd} returns the internal \syscall{kqueue}, \syscall{epoll}, or Ports descriptor; \method{:events} returns ``r''; and \method{:timeout} returns the time to the next internal timeout event.} In particular, this means that you can stack \cqueues, or poll on a \cqueues object using some other event loop library. Each \cqueues object is entirely self-contained, without any global state.

\subsection{$\lnot$ Globals}

Like the core controller module, other \cqueues modules adhere to a no-global-side-effects discipline. In particular, this means
\begin{itemize}
\item no global process variables;
\item no signal handling gimmicks---like the pipe trick---which could conflict with other components of your application\footnote{The \module{cqueues.thread} module ensures threads are started with a filled signal mask.}; and
\item consistent use of O\_CLOEXEC and similar syscall flags to eliminate or reduce fork+exec races in threaded applications.
\end{itemize}

\subsection{Errors}

The usual behavior is for errors to be returned directly. But see \routine{socket.onerror}. If a routine is specified to return an object or string, nil is returned; if a boolean, false is returned. In both cases, these  are usually followed by a numeric error code. Thus, if a routine is specified to return two values on success, then on error three values are returned, the first two nil or false, and the third an error code.

\cqueues is a relatively low-level component library. In almost all cases errors will be system errors, returned as numeric error codes for easy and efficient comparison. For example, attempting to create a UNIX domain socket with \routine{socket.listen} in a directory without sufficient permissions might return `nil, \errno{EACCES}'.

\subsubsection{\texttt{EAGAIN}}

\cqueues modules are implemented in both C and Lua. The C routines never yield, and always return recoverable errors directly. Most C routines are wrapped---and methods interposed---with Lua functions. These Lua functions usually poll when \errno{EAGAIN} is encountered and retry the C routine on resumption. Few methods will return \errno{EAGAIN} directly.

\subsubsection{\texttt{ETIMEDOUT}}

This error value is usually seen when a timeout is specified by the caller of a blocking method. The method will normally poll if the operation cannot be completed immediately, but if the timeout expires then it will return a failure with \errno{ETIMEDOUT}.

\subsubsection{\texttt{EPIPE}}

In Unix \errno{EPIPE} is only encountered when attempting to write to a closed pipe or socket. In \cqueues \errno{EPIPE} is used to signal both EOF and a closed output stream.\footnote{In some situations, such as with SSL/TLS, a read attempt might require a write, anyhow. Expanding the scope of EPIPE simplifies the logic required to handle various I/O failures.} The low-level I/O method \method{socket:recv}, for example, returns \errno{EPIPE} on EOF. In other cases, as with \method{socket:read}, EOF is not an error condition.

\subsubsection{\texttt{EBADF}}

This error commonly occurs in buggy asynchronous applications, which juggle descriptors and descriptor state. In Lua code using the \cqueues APIs, \errno{EBADF} should never be encountered. When it does occur, it's a sure sign of a bug somewhere.

\section{Modules}

\subsection{\cqueues}

\subsubsection{\routine{cqueues.VENDOR}}
String describing the vendor, e.g.\ william@25thandClement.com. If you fork this project please change this string so I don't receive unwarranted scorn or praise.

\subsubsection{\routine{cqueues.VERSION}}
Number describing the running version, formatted as YYYYMMDD. Official releases are tagged in the git repo as rel-YYYYMMDD.

\subsubsection{\routine{cqueues.COMMIT}}
Git commit hash string of HEAD.

\subsubsection[\routine{cqueues.interpose}]{\routine{cqueues.interpose(name, function)}}
Add or interpose a \cqueues controller class method. Returns the previous method, if any.

\subsubsection[\routine{cqueues.monotime}]{\routine{cqueues.monotime()}}
Return the system's monotonic clock time, usually clock\_gettime(CLOCK\_MONOTONIC).

\subsubsection[\routine{cqueues.cancel}]{\routine{cqueues.cancel(fd)}}
Cancels the specified descriptor for all controllers. This ensures safe early closure of descriptors. However, the complexity is approximately M 2 log N, where M is the number of controllers, and N the number of descriptors per controller (presuming equal distribution). For most purposes this is entirely inconsequential. By contrast, however, implicit cancellation through GC or yield/resume cycling is O(1).

Any coroutine polling on the canceled descriptor is placed on its controller's pending queue.

\subsubsection[\routine{cqueues.poll}]{\routine{cqueues.poll($\ldots$)}}
Takes a series of objects obeying the polling protocol and yields control to the parent \cqueues controller. On an event resumes the coroutine, passing the objects which triggered resumption. A number value is interpreted as a timeout.

\subsubsection[\routine{cqueues.sleep}]{\routine{cqueues.sleep(number)}}

Yields to the parent \cqueues controller and schedules a wakeup for `number' seconds in the future.

\subsubsection[\routine{cqueues.new}]{\routine{cqueues.new()}}
Create a new cqueues object.

\subsubsection[\routine{cqueues:attach}]{\routine{cqueue:attach(coroutine)}}
Attach and manage the specified coroutine.

\subsubsection[\routine{cqueues:wrap}]{\routine{cqueue:wrap(function)}}
        Execute function inside a new coroutine managed by the controller.

\subsubsection[\routine{cqueues:step}]{\routine{cqueue:step([timeout])}}
Step once through the event queue. If no timeout is provided, it blocks indefinitely until a descriptor event or timeout fires.

Returns true or false. If false, then a second return value holds the error message. :step() can be called again after errors, for example if a coroutine threw an error.
%[However, at the moment the controller cannot recover from synchronization errors with the kernel queue. In the future, internal, unrecoverable errors should be thrown and only coroutine errors returned directly.]

\subsubsection[\routine{cqueues:empty}]{\routine{cqueue:empty()}}
Returns true if there are no more descriptor or timeout events queued, false otherwise.

\subsubsection[\routine{cqueues:count}]{\routine{cqueue:count()}}
Returns a count of managed coroutines.

\subsubsection[\routine{cqueues:cancel}]{\routine{cqueue:cancel(fd)}}
Cancel the specified descriptor for that controller. See cqueues.cancel.

\subsubsection[\routine{cqueues:pause}]{\routine{cqueue:pause(signal [, signal $\ldots$ ])}}
A wrapper around \syscall{pselect} which suspends execution until the controller polls ready or a signal is delivered. This interface is provided as a very basic least common denominator for simple slave process controller loops and similar scenarios, where immediate response to signal delivery is required on platforms like Solaris without a proper signal polling primitive. (\routine{signal.listen} on Solaris merely periodically queries the pending set.)

Much better alternatives are possible for Solaris, but require global process state or an LWP thread helper.

\subsection{\module{cqueues.socket}}

\subsubsection[\fn{socket.interpose}]{\fn{socket.interpose(name, function)}}
Add or interpose a socket class method. Returns the previous method, if any.

\subsubsection[\fn{socket.connect}]{\fn{socket.connect(host, port [, mode])}}
Return a new socket immediately ready for reading or writing. DNS lookup and TCP connection handling are handled transparently.

`mode' is a string containing one or more of the following flags

\begin{ctabular}{c | p{6in}}
flag & description \\\hline
t & text mode; input and output undergo LF/CRLF translation \\
b & binary mode; no LF/CRLF translation \\
n & no output buffering \\
l & line buffered output \\ 
f & fully buffered output \\
\end{ctabular}
At present the default mode is ``tl''---text translation and line buffering. This may change depending on how much confusion it creates, perhaps to ``tn''.

\subsubsection[\fn{socket.connect}]{\fn{socket.connect\{ $\ldots$ \}}}
Like socket.connect, but takes a table of options:

\begin{ctabular}{r | c | p{4.5in}}
field & type:default & description\\\hline
.host & string:nil & IP address or host domain name \\

.port & string:nil & host port \\

.path & string:nil & UNIX domain socket path \\

.mode & string:nil & fchmod or chmod socket after creating UNIX domain socket
%NOTE: There's a race between bind and the following chmod. fchmod is attempted before the bind, however it fails on BSD derivatives. Not all platforms obey UNIX domain socket permissions (e.g. Solaris). Check peer credentials, instead, to be portable.
\\

.mask & string:nil & set and restore umask when binding UNIX domain sockets %NOTE: Not all platforms obey UNIX domain socket permissions. Check peer credentials, instead, to be portable.
\\

.unlink & boolean:false & unlink socket path before binding \\

.reuseaddr & boolean:true & SO\_REUSEADDDR socket option \\

.nodelay & boolean:false & TCP\_NODELAY IP option \\

.nopush & boolean:false & TCP\_NOPUSH, TCP\_CORK, or equivalent IP option \\

.nonblock & boolean:true & O\_NONBLOCK descriptor flag \\

.cloexec & boolean:true & O\_CLOEXEC descriptor flag \\

.nosigpipe & boolean:true & O\_NOSIGPIPE, SO\_NOSIGPIPE, MSG\_NOSIGNAL, or equivalent descriptor flag \\

.verify & boolean:false & require SSL certificate verification (UNFINISHED) \\

.time & boolean:true & track elapsed time for statistics \\
\end{ctabular}

\subsubsection[\fn{socket.listen}]{\fn{socket.listen(host, port)}}
	Return a new socket immediately ready for accepting connections.

\subsubsection[\fn{socket.listen}]{\fn{socket.listen\{ $\ldots$ \}}}
	Like socket.listen. See socket.connect\{\}.

\subsubsection[\fn{socket.pair}]{\fn{socket.pair([type])}}
	Return two bound sockets. Type is either "stream" or "dgram", and
	defaults to "stream".

\subsubsection[\fn{socket.setvbuf}]{\fn{socket.setvbuf(mode [, size])}}
	Set the default output buffering. See socket:setvbuf.

\subsubsection[\fn{socket.setmode}]{\fn{socket.setmode([input], [output])}}
	Set the default I/O mode. See socket:setmode.

\subsubsection[\fn{socket.onerror}]{\fn{socket.onerror([function])}}
	Set the default error handler. See socket:onerror.

%\subsubsection[\fn{socket:connect}]{\fn{socket:connect()}}
%	Wait for connection establishment to succeed. [UNFINISHED]

%\subsubsection[\fn{socket:listen}]{\fn{socket:listen()}}
%	Wait for socket binding to complete, which may have been delayed if
%	the specified host required DNS resolution. [UNFINISHED]

\subsubsection[\fn{socket:accept}]{\fn{socket:accept([timeout])}}
	Wait for and return an incoming client socket on a listening object.

\subsubsection[\fn{socket:clients}]{\fn{socket:clients([timeout])}}
	Iterator over socket:accept: for con in srv:clients() do ... end.

%\subsection[\fn{socket:certify}]{\fn{socket:certify(certificate)}}
%	Associate a certificate for subsequent :starttls operation.
%	[UNFINISHED]

\subsubsection[\fn{socket:starttls}]{\fn{socket:starttls()}}
	Place socket into TLS mode. Returns immediately.
%[FUTURE: take parameter to return immediately or wait for channel establishment.]

\subsubsection[\fn{socket:setvbuf}]{\fn{socket:setvbuf(mode [, size])}}
	Same as Lua file:setvbuf. Analagous to "n", "l", and "f" mode flags.

\subsubsection[\fn{socket:setmode}]{\fn{socket:setmode([input], [output])}}
	See socket.connect for mode flags. One or both modes can be nil,
	in which case the mode is left unchanged.

	Returns the previous input and output modes as fixed-sized strings.
	At present the first character is one of "t" or "b", and the second
	character one of "n", "l", "f", or "-" (for in the input mode).

\subsubsection[\fn{socket:onerror}]{\fn{socket:onerror([function])}}
	Set the error handler. The error handler is passed the tuple
	socket-object, method-string, error-number, and is expected to
	either throw an error or return an error-number--to be returned to
	the caller as part of the documented return interface.

	The default error handler returns EPIPE and ETIMEDOUT directly, and
	throws everything else. EAGAIN is handled internally for blocking
	calls.

	Returns the previous error handler, if any.

\subsubsection[\fn{socket:read}]{\fn{socket:read(...)}}
	Similar to Lua's \fn{file:read}, with additional formats.

\begin{tabular}{c | l}
format & description\\\hline
{*n} & unsupported \\
{*a} & unsupported \\
{*l} & read the next line, trimming the EOL marker \\
{*L} & read the next line, keeping the EOL marker \\
number & read `number' bytes or until EOF \\
{*h} & read and unfold MIME compliant header \\
{*H} & read MIME compliant header, keeping EOL markers \\
-number & read 1 to `number' bytes, immediately returning if possible \\
\end{tabular}

\subsubsection[\fn{socket:write}]{\fn{socket:write(...)}}
	Same as Lua \fn{file:write}.

\subsubsection[\fn{socket:flush}]{\fn{socket:flush([mode])}}
Flushes output buffer. Mode is one of the ``nlf'' flags described in \method{socket.connect}. A nil mode implies ``n'', i.e.\ no buffering and  effecting a full flush. An empty string mode resolves to the configured output buffering mode.

\subsubsection[{\fn{socket.unget}}]{\fn{socket:unget(string)}}
Writes `string' to the head of the socket input buffer.

\subsubsection[{\fn{socket.pending}}]{\fn{socket:pending()}}
Returns as two values the count of buffered bytes in the input and output streams.

\subsubsection[\fn{socket:uncork}]{\fn{socket:uncork()}}
	Disables TCP\_NOPUSH, TCP\_CORK, or equivalent socket option.

\subsubsection[\fn{socket:recv}]{\fn{socket:recv(format [, mode])}}
Similar to \method{socket:read}, except takes only a single format and returns immediately without polling. On success returns the string or number. On failure returns nil and a numeric error code--usually EAGAIN or EPIPE. Does not use error handler.

`mode' is as described in \fn{socket.connect}, and defaults to the configured input mode.

\subsubsection[\fn{socket:send}]{\fn{socket:send(string, i, j [, mode])}}
Write out the slice `string'[i,j]. Similar to passing \fn{string:sub(i, j)}, but without instantiating a new string object. Immediately returns two values: count of bytes written (0 to j-i+1), and numerical error code, if any (usually EAGAIN or EPIPE).

\subsubsection[\fn{socket:recvfd}]{\fn{socket:recvfd([prepbufsiz])}}
Receive an ancillary socket message with accompanying descriptor. `prepbufsiz' specifies the maximum message size to expect.

This routine bypasses I/O buffering.

Returns message-string, socket-object on success; nil, nil, error-integer on failure. On success socket-object may still be nil. Message truncation is treated as an error condition.

\subsubsection[\fn{socket:sendfd}]{\fn{socket:sendfd(msg, socket)}}
	Send an ancillary socket message with accompanying descriptor. msg
	should be a non-zero-length string, which some platforms require.
	socket should be a Lua file handle, \cqueues socket, integer
	descriptor, or nil.

	This routine bypasses I/O buffering.

	Returns true on success; false, error-integer on failure.

\subsubsection[\fn{socket:shutdown}]{\fn{socket:shutdown(how)}}
Simple binding to \syscall{shutdown(2)}. `how' is a string containing one or both of the flags `r' or `w'.

\begin{tabular}{r | l}
`r' & analagous to \syscall{shutdown(SHUT\_RD)} \\
`w' & analagous to \syscall{shutdown(SHUT\_WR)} \\
\end{tabular}

\subsubsection[\fn{socket:eof}]{\fn{socket:eof()}}
Returns 2 boolean values representing whether EOF has been received on the input channel, and whether the output channel has signaled closure (e.g. EPIPE).

\subsubsection[\fn{socket:close}]{\fn{socket:close()}}
Explicitly and immediately close all internal descriptors. This routine ensures all descriptors are properly cancelled.

\subsection{\module{cqueues.errno}}

\subsubsection[\fn{errno[]}]{\fn{errno[]}}
A table mapping all system error string macros to numerical error codes, and all numerical error codes to system error string macros. Thus, errno.EAGAIN evaluates to a numeric error code, and errno[errno.EAGAIN] evaluates to the string ``EAGAIN''.

\subsubsection[\fn{errno.strerror}]{\fn{errno.strerror(code)}}
Returns string returned by strerror(3). [FUTURE: Will also handle embedded socket.c and dns.c library error codes.]

\subsection{\module{cqueues.signal}}

\subsubsection{\fn{signal[]}}
A table mapping signal string macros to numerical signal codes.
This, signal.SIGKILL likely evaluates to the number 9.

\subsubsection[\fn{signal.strsignal}]{\fn{signal.strsignal(code)}}
Returns string returned by strsignal(3).

\subsubsection[\fn{signal.ignore}]{\fn{signal.ignore(signal [, signal $\ldots $ ])}}
Set the signal handler to SIG\_IGN for the specified signals.

\subsubsection[\fn{signal.default}]{\fn{signal.default(signal [, signal $\ldots$ ])}}
Set the signal handler to SIG\_DFL for the specified signals.

\subsubsection[\fn{signal.discard}]{\fn{signal.discard(signal [, signal $\ldots$ ])}}
Set the signal handler to a builtin ``noop'' handler for the specified signals. Use this is you want signals to interrupt syscalls.

\subsubsection[\fn{signal.block}]{\fn{signal.block(signal [, signal $\ldots$ ])}}
Block the specified signals.

\subsubsection[\fn{signal.unblock}]{\fn{signal.unblock(signal [, signal $\ldots$ ])}}
Unblock the specified signals.

\subsubsection[\fn{signal.raise}]{\fn{signal.raise(signal [, signal $\ldots$ ])}}
raise(3) the specified signals.

\subsubsection[\fn{signal.interpose}]{\fn{signal.interpose(name, function)}}
Add or interpose a signal listener class method. Returns the previous method, if any.

\subsubsection[\fn{signal.listen}]{\fn{signal.listen(signal [, signal $\ldots$ ])}}
Returns a signal listener object for the specified signals. Semantics differ between platforms:

\paragraph{kqueue}
BSD \syscall{kqueue} provides the most intuitive and desirable behavior. All listeners will detect a signal sent to the process irrespective of whether the signal is ignored, blocked, or delivered. However, EVFILT\_SIGNAL is edge-triggered, which means no notification of subsequent delivery of a pending signal which has been unblocked.

\paragraph{signalfd}
Linux \syscall{signalfd} will not detect ignored or delivered signals, and only one signalfd object will poll ready per signal.

\paragraph{sigtimedwait}
Solaris provides no signal polling kernel primitive. Instead, the pending set is periodically queried using \syscall{sigtimedwait}. See \method{signal:settimeout}. Like Linux, only one listener can notify per interrupt.

To be portable the application must block the relevant signals. See signal.block. Otherwise, neither Linux nor Solaris will be able to detect the interrupt. Any signal should be assigned to one listener only, although any listener may query multiple signals.

Alternatively, applications may start a dedicated thread to field incoming signals, and send notifications over a socket. In the future this may be provided as an optional listener implementation.

See also \routine{cqueue:pause} for another, if crude, alternative.

\subsubsection[\fn{signal:wait}]{\fn{signal:wait([timeout])}}
Polls for the signal set passed to the constructor. Returns the signal number, or nil on timeout.

\subsubsection[\fn{signal:settimeout}]{\fn{signal:settimeout(timeout)}}
Set the polling period for implementations such as Solaris which lack a signal polling kernel primitive. On such systems signal:wait merely queries the pending set every `timeout' seconds.

\subsection{\module{cqueues.thread}}

\subsubsection[\fn{thread.self}]{\fn{thread.self()}}
Returns the LWP thread object for the running Lua instances. Threads not started via thread.start return nil.

\subsubsection[\fn{thread.start}]{\fn{thread.start(function [, string [, string $\ldots$ ]])}}
Generates a socket pair, starts a POSIX LWP thread, initializes a new Lua VM instance, preloads the \cqueues library, and loads and executes the specified function from the new LWP thread and Lua instance. The function receives as the first parameter one end of the socket pair---instantiated as a cqueues.socket object---followed by the string parameters passed to thread.start.

The new LWP thread starts with all signals blocked.

Returns a thread object and a socket object---the other end of the socket pair. On error returns two nils and an error code.

\subsubsection[\fn{thread:join}]{\fn{thread.join([timeout])}}
Wait for the thread to terminate. Calling the equivalent of thread.self():join() is disallowed.

Returns a boolean and error value. If false, error value is an error code describing a local error, usually EAGAIN or ETIMEDOUT. If true, error value is 1) an error code describing a system error which the thread encountered, 2) an error message string returned by the new Lua instance, or 3) nil if completed successfully.

\subsection{\module{cqueues.notify}}

\subsubsection[\fn{notify[]}]{\fn{notify[]}}
A table mapping bitwise flags to names.
\begin{ctabular}{r | p{6in}}

name & description \\\hline
CREATE & file creation event \\
DELETE & file deletion event \\
ATTRIB & metadata change event \\
MODIFY & modification to file contents or directory entries \\ 
REVOKE & permission revoked
%\hline
%INOTIFY   & foo \\
%FEN       & foo \\
%KQUEUE    & foo \\
%KQUEUE1   & foo \\
%OPENAT    & foo \\
%FDOPENDIR & foo \\
%O\_CLOEXEC & foo \\
%IN\_CLOEXEC & foo \\
\end{ctabular}

\subsubsection[\fn{notify.opendir}]{\fn{notify.opendir(path[, changes ])}}

Returns a notification object associated with the specified directory. Directory change events are limited to the set, `changes', or to notify.ALL if nil.

\subsubsection[\fn{notify:add}]{\fn{notify:add(name[, changes ])}}

Track the specified file name within the notification directory. `changes' defaults to notify.ALL if nil.

\subsubsection[\fn{notify:get}]{\fn{notify:get([timeout])}}

Returns a bitwise change set and a filename on success.


\appendix
\printindex

\end{document}
